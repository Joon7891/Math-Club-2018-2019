\documentclass[12pt]{article}
\usepackage[a4paper, total={7in, 10in}]{geometry}
\usepackage{amsmath}
\usepackage{amssymb}
\usepackage{graphicx}

\title{Week 2 Problem Set Solutions\vspace{-3mm}}
\author{2018-2019 SLSS Math Club\vspace{-5mm}}
\date{October 23, 2018 \vspace{-5mm}}

\begin{document}
\maketitle

\section*{Basic Problems}
\begin{enumerate}
    \item \textbf{A function $f$ satisfies: \\ $$f(x) + f(x + 3) = 2x + 5$$ \\ If $f(2) + f(8) = 12$, what is the value of $f(5)$?} \\
    
    Set $x = 2$ and $x = 5$ and add the equations to obtain the following:
    \begin{align*}
        f(5)+f(8)&=(2)(5)+5 \\
        + f(2)+f(5)&=(2)(2)+5 \\[-10pt]
        \cline{1-2}
        2f(5)+f(2)+f(8)&=15+9 \\
        2f(5)+12&=24 \\
        2f(5)&=12 \\
        f(5)&=6
    \end{align*}
    
    $\therefore f(5) = 6$

    \item \textbf{If $f(x+1)=\frac{2f(x)+1}{3}$ and $f(2) = 2$, find the value of $f(1)$.} \\
    
    If $x = 1$,
    \begin{align*}
        f(x+1)&=\frac{2f(x)+1}{3} \\
        f(1 + 1)&=\frac{2f(1)+1}{3} \\
        f(1)&=\frac{3f(2)-1}{2} \\
        f(1)&=\frac{3(2)-1}{2} \\
        f(1)&=2.5
    \end{align*}
    
    $\therefore f(1) = 2.5$
    
\end{enumerate} \newpage
\section*{Intermediate Problems}
\begin{enumerate}
    \item \textbf{The function $f(x)$ has the following property: $$f(x+y)=f(x)+f(y)+2xy$$ for all positive integers $x$ and $y$. If $f(1)=4$, determine $f(8)$.} \\
    
    Let $y = x$:
    \begin{align*}
        f(x + x) &= f(x) + f(x) + 2xx \\
        f(2x) &= 2f(x) + 2x^2
    \end{align*}
    Solving for $f(8)$:
    \begin{align*}
        f(2 \cdot 1) &= 2f(1) + 2(1)^2 \\
        f(2) &= 2(4) + 2 \\
        f(2) & = 10 \\ \\
        f(2 \cdot 2) &= 2f(2) + 2(2)^2 \\
        f(4) &= 2(10) + 8 \\
        f(4) & = 28 \\ \\
        f(2 \cdot 4) &= 2f(4) + 2(4)^2 \\
        f(8) &= 2(28) + 32 \\
        f(8) & = 88
    \end{align*}
    
    $\therefore f(8) = 88$

    \item \textbf{Let $f$ be a real valued function such that: $$f(m + n) = f(m)\cdot f(n)$$ If $f(4) = 256$ and $f(k) = 0.0625$, find the value of $k$.}
    \begin{align*}
        f(4+0)&=f(4)\cdot f(0) \\
        f(0)&=1 \\ \\
        f(1+1)&=f^2(1) \\
        \therefore f(2) = f^2(1) & \geq 0 \\ \\
        f(2+2)&=f^2(2)=256 \\
        f(2)&=16 \\ \\
        f(2-2)&=f(2)\cdot f(-2)=1 \\
        f(-2)&=\frac{1}{16}=0.0625 \\
        k&=-2
    \end{align*}
    
\end{enumerate}

\section*{Advanced Problem}
\begin{enumerate}
    \item \textbf{A function is defined for integers $a$ and $b$ as follows: $$f(ab) = f(a)f(b) - f(a + b) + 1989$$ where either $a$ or $b$ is $1$, and $f(1) = 2.$
    \begin{enumerate}
        \item Prove that $f(n) = f(n - 1) + 1989$ \\ \\
        {\normalfont Set $a=1$ and $b=n-1$}
        \begin{align*}
            f(n-1)&=f(1)f(n-1)-f(n)+1989 \\
            f(n-1)&=2f(n-1)-f(n)+1989 \\
            f(n)&=f(n-1)+1989
        \end{align*} %Adding whitespace here seems to break the compilation
        \item Determine the value of $f(2001)$ \\ \\
        {\normalfont Since $f(n)-f(n-1)=1989$, then $f(n)$ is an arithmetic sequence with the first term $f(1)=2$ and common difference 1989.}
        \begin{align*}
            f(2001)&=2+2000(1989) \\
            &=3978002
        \end{align*}
    \end{enumerate}}
    
    \item \textbf{Let $f$ be a real valued function such that: $$f(x) + xf(1 - x) = 1 + x^2$$ for all real $x$. Determine $f(x)$.} \\ \\
    Set $x = 1-x$
    \begin{align*}
        f(1-x)+(1-x)f(x)&=1+(1-x)^2 \\
        f(1-x)+(1-x)f(x)&=x^2-2x+2 \\
    \end{align*}
    To remove $f(1 - x)$ we multiply the second equation by $x$ and subtraction from the original equation
    \begin{align*}
        f(x)+x(1-x)-xf(1-x)-x(1-x)f(x)&=1+x^2-2x+2x^2-x^3 \\
        (1-x+x^2)f(x)&=1-2x+3x^2-x^3 \\
        f(x)=\frac{1-2x+3x^2-x^3}{1-x+x^2}
    \end{align*}
    \newpage
    
    \item \textbf{If $f(x)=3x^2-2x+5$ and $f(g(x))=12x^4+56x^2+70$, find all possible values for the sum of coefficients of $g(x)$.} \\
    
    The sum of the coefficients of $g(x)$ is obtained by finding $g(1)$
    \begin{align*}
        f[g(x)] &= 3[g(x)]^2 - 2[g(x)] + 5 \\
        f[g(1)] &= 3[g(1)]^2 - 2[g(1)] + 5 \\
        &= 12 + 56 + 70 \\
        0 &=3[g(1)]^2 - 2[g(1)] - 133 \\
        0 &= [3g(1) + 19][g(1) - 7]
    \end{align*}
    $\therefore$ The sum of the coefficients of $g(x)$ is $-\frac{19}{3}$ or $7$
\end{enumerate}

\end{document}