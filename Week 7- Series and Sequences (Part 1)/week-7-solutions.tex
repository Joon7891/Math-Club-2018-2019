\documentclass[12pt]{article}
\usepackage[a4paper, total={7in, 10in}]{geometry}
\usepackage{amsmath}
\usepackage{graphicx}

\title{Week 7 Problem Set Solutions\vspace{-3mm}}
\author{2018-2019 SLSS Math Club\vspace{-5mm}}
\date{November 27, 2018\vspace{-5mm}}

\begin{document}
\maketitle

\newcommand{\bspace}{\\ \\ \\ \\ \\ \\ \\} 
\section*{Basic Problems}
\begin{enumerate}
    \item \textbf{Express $1 + 2 + 3 + 4 + 5 + \dots + n$ in sigma notation.} \\ \\
    From observation, the explicit formula is $a_i = i$, with $1$ being the starting point and $n$ being the ending point. \\
    
    This gives us the following sigma notation expression:
    \begin{equation*}
        1 + 2 + 3 + 4 + 5 + \dots + n = \sum_{i = 1}^n i
    \end{equation*}
    
    \item \textbf{Express $1 + 3 + 7 + 15 + 31 + 63 + 127 + 255 + 511 + 1023$ in sigma notation.} \\
    
    The explicit formula for this sequence is $a_n = 2^n - 1$. The starting point is $1$ and the ending point is $10$; given that there are $10$ terms in the sequence. \\
    
    This gives us the following sigma notation expression:
    \begin{equation*}
        1 + 3 + 7 + 15 + 31 + 63 + 127 + 255 + 511 + 1023 = \sum_{n = 1}^{10} (2^n - 1)
    \end{equation*}
    
    \item \textbf{In an arithmetic sequence $a_1 = 83$ and $a_4 = 98$, determine $a_n$.} \\
    
    We know that:
    \begin{equation*}
        a_n = a_n + d(n - 1)        
    \end{equation*}
    
    Using $a_1$ and $a_4$ to determine $d$:
    \begin{align*}
        a_n & = a_1 + d(n-1) \\
        a_4 & = a_1 + d(4 - 1) \\
        98 & = 83 + 3d \\
        3d & = 15 \\
        d & = 5
    \end{align*}
    
    Using $d$ and $a_1$ to determine $a_n$:
    \begin{align*}
        a_n & = a_1 + d(n-1) \\
        a_n & = 83 + 5(n-1) \\
        a_n & = 78 + 5n
    \end{align*} 
    Therefore, $a_n = 5n + 78$
    
    \newpage
    
    \item \textbf{In a geometric sequence $a_5 = 32$ and $r = \frac{1}{2}$, determine $a_n$.} \\
    
    We know that: 
    \begin{equation*}
        a_n = a_1(r)^{n - 1}
    \end{equation*}
    Using $a_5$ and $r$ to determine $a_1$:
    \begin{align*}
        a_n & = a_1 (r)^{n - 1} \\
        32 & = a_1(\frac{1}{2})^{5 - 1} \\
        32 & = a_1(\frac{1}{2})^4 \\
        a_1 & = 32 (2)^4 \\
        a_1 & = 512
    \end{align*}
    Using values for $a_1$ and $r$, we obtain the following:
    \begin{align*}
        a_n & = a_1(r)^{n - 1} \\
        a_n & = 512(\frac{1}{2})^{n - 1}
    \end{align*}
    
    Therefore, $a_n = 512(\frac{1}{2})^{n - 1}$
\end{enumerate}

\section*{Intermediate Problems}
\begin{enumerate}
    \item \textbf{Determine the sum of all even numbers between $1001$ and $1801$.} \\
    
    Note: 
    \begin{itemize}
        \item The sum of the first $n$ positive integers is $\frac{n(n+1)}{2}$
        \item The first even number in the given range is $1002$ while the last even number is $1800$.
    \end{itemize}
    Let $S$ be the sum of all even numbers between $1001$ and $1801$:
    \begin{align*}
        S & = 1002 + 1004 + 1006 + \dots + 1796 + 1798 + 1800 \\
        S & = 2(501 + 502 + 503 + \dots + 898 + 899 + 900) \\
        S & = 2\sum_{n = 501}^{900} n \\
        S & = 2(\sum_{n = 1}^{900} n - \sum_{n = 1}^{500}n) \\
        S & = 2[\frac{(900)(901)}{2} - \frac{(500)(501)}{2}] \\
        S & = 560,400
    \end{align*}
    Therefore, the sum of all even numbers between $1001$ and $1801$ is $560,400$.
    
    \newpage
    
    \item \textbf{In a geometric sequence, $a_2 = 2,$ and $r > 0$. If $a_3a_9 = 2a^2_5$, what is the value of $a_1$?} \\
    
    In order to solve for $r$, we should consider writing $a_3, a_9,$ and $a_5$ in terms of $a_2$. \\
    
    We know that:
    \begin{align*}
        a_3 & = a_2r \\
        a_5 & = a_2r^3 \\
        a_9 & = a_2r^7
    \end{align*}
    
    Substituting $a_3, a_5,$ and $a_9$ to solve for $r$:
    \begin{align*}
        a_3a_9 & = 2a_5^2 \\
        (a_2r)(a_2r^7) & = 2(a_2r^3)^2 \\
        a_2^2r^8 & = 2a_2^2r^6 \\
        r^2 & = 2 \\
        r & = \pm \sqrt{2}
    \end{align*}
    
    As $r>0$, $r = \sqrt{2}$. \\
    
    Solving for $a_1$:
    \begin{align*}
        a_2 & = a_1r \\
        a_1 & = \frac{a_2}{r} \\
        a_1 & = \frac{2}{\sqrt{2}} \\
        a_1 & = \sqrt{2}
    \end{align*}
    
    Therefore, $a_1 = \sqrt{2}$ \\
    
    \item \textbf{In an arithmetic sequence, the sum of the first $n$ terms of an arithmetic sequence is $S_n = 4n^2 - n + 2$, what is $a_n$?} \\
    
    We know that:
    \begin{align*}
        S_n &= a_1 + a_2 + \dots + a_{n - 1} + a_n \\
        S_{n - 1} &= a_n + a_2 + \dots + a_{n - 1}
    \end{align*}
    
    Substituting $S_{n - 1}$ into $S_n$, we obtain the following:
    \begin{align*}
        S_n & = a_1 + a_2 + \dots + a_{n - 1} + a_n \\
        S_n & = S_{n - 1} + a_n \\
        a_n & = S_n - S_{n - 1} \\
        a_n & = 4n^2 - n + 2 - [4(n - 1)^2 - (n - 1) + 2] \\
        a_n & = 8n - 5
    \end{align*}
    Therefore, $a_n = 8n - 5$
\end{enumerate}

\newpage
\section*{Advanced Problems}
\begin{enumerate}
    \item \textbf{Determine the sum of the following sequence:
    \begin{align*}
        S & = \frac{1}{2} + \frac{1}{6} + \frac{1}{12} + \frac{1}{20} + \dots + \frac{1}{10100}
    \end{align*}} \\
    
    Given the large amount of terms in the series, a telescoping sum is to be taken, meaning the explicit formula $a_n$ must be constructed to have two or more terms. \\
    
    Given that:
    \begin{align*}
        \frac{1}{2} = \frac{1}{1(2)} \\
        \frac{1}{6} = \frac{1}{2(3)} \\
        \frac{1}{12} = \frac{1}{3(4)}
    \end{align*}
    we can state that $a_n = \frac{1}{n(n+1)}$, which we can decompose into $\frac{A}{n} + \frac{B}{n - 1}$. \\
    
    Decomposing explicit formula:
    \begin{align*}
        \frac{1}{n(n+1)} & = \frac{A}{n} + \frac{B}{n + 1}\\
        1 & = A(n+1)+B(n) \\
        1 & = (A + B)n + A
    \end{align*}
    
    By comparing coefficients we obtain the following:
    \begin{align*}
        A & = 1 \\ 
        B & = -1 \\
        \frac{1}{n(n+1)} & = \frac{1}{n} - \frac{1}{n + 1}
    \end{align*}
    
    Using this formula, we obtain the following:
    \begin{align*}
        S & = \frac{1}{2} + \frac{1}{6} + \frac{1}{12} + \frac{1}{20} + \dots + \frac{1}{10100} \\
        S & = \frac{1}{1(2)} + \frac{1}{2(3)} + \frac{1}{3(4)} + \frac{1}{4(5)} + \dots + \frac{1}{100(101)} \\
        S & = (\frac{1}{1} - \frac{1}{2}) +  (\frac{1}{2} - \frac{1}{3}) +  (\frac{1}{3} - \frac{1}{4}) +  (\frac{1}{4} - \frac{1}{5}) + \dots + (\frac{1}{100} - \frac{1}{101})\\
        S & = 1 - \frac{1}{101} \\
        S & = \frac{100}{101}
    \end{align*}
    
    Therefore, $S = \frac{100}{101}$.
    
    \item \textbf{Define $f(x)=\frac{x^2}{1+x^2}$ and suppose that \begin{align*}
        A & = f(1) + f(2) + f(3) + \dots + f(2018) \\
        B & = f(1) + f(\frac{1}{2}) + f(\frac{1}{3}) + \dots + f(\frac{1}{2018})
    \end{align*}
    Determine the value of $A + B$} \\
    
    Given that $A + B = f(1) + f(\frac{1}{1}) + f(2) + f(\frac{1}{2}) + f(3) + f(\frac{1}{3}) + \dots + f(2018) + f(\frac{1}{2018})$, we should consider a function $g(x) = f(x) + f(\frac{1}{x})$. \\
    
    Manipulating $g(x)$:
    \begin{align*}
        g(x) & = f(x) + f(\frac{1}{x}) \\
        g(x) & = \frac{x^2}{1 + x^2} + \frac{\frac{1}{x^2}}{1 + \frac{1}{x^2}} \\
        g(x) & = \frac{x^2}{1 + x^2} + \frac{1}{x^2 + 1} \\
        g(x) & = \frac{x^2 + 1}{x^2 + 1} \\
        g(x) & = 1
    \end{align*}
    
    Solving for $A + B$:
    \begin{align*}
        A + B & = \sum_{x = 1}^{2018} f(x) + \sum_{x = 1}^{2018} f(\frac{1}{x}) \\
        A + B & = \sum_{x = 1}^{2018} [f(x) + f(\frac{1}{x})] \\
        A + B & = \sum_{x = 1}^{2018} g(x) \\
        A + B & = \sum_{x = 1}^{2018} 1 \\
        A + B & = 2018
    \end{align*}
    
    Therefore, $A + B = 2018$.
    
\end{enumerate}

\end{document}