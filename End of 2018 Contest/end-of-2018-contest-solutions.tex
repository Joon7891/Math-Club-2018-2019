\documentclass[12pt]{article}
\usepackage[a4paper, total={7in, 10in}]{geometry}
\usepackage{amsmath}
\usepackage{graphicx}

\title{End of 2018 Contest Solutions\vspace{-3mm}}
\author{2018-2019 SLSS Math Club\vspace{-5mm}}
\date{January 7, 2019\vspace{-5mm}}

\begin{document}
\maketitle

\section*{Problems}

\begin{enumerate}
    \item \textbf{Given that the sums of the reciprocals of the roots of the quadratic $$ 0 = 3x^2 + 7x + k$$ is $\displaystyle{\frac{7}{3}}$, determine $k$. [5 Marks]} \\
    
    Let $r1$ and $r2$ represent the roots of the above quadratic.  From Vieta's formulas, we know the following:
    \begin{align*}
        r_1r_2 &= \frac{k}{3} \\
        r_1 + r_2 &= -\frac{7}{3}
    \end{align*}
    
    Rearranging $\frac{1}{r_1} + \frac{1}{r_2}$ and substituting the above equations to solve for $k$, we yield the following:
    \begin{align*}
        \frac{1}{r_1} + \frac{1}{r_2} &= \frac{7}{3} \\
        \frac{r_1 + r_2}{r_1r_2} &= \frac{7}{3} \\
        -\frac{7}{k} &= \frac{7}{3} \\
        k &= -3
    \end{align*}
    
    Therefore, $k = -3$.
    
    \item \textbf{Given that $$\log_2(x), 1 + \log_4(x), \log_8(4x)$$ are consecutive terms of a geometric sequence, determine all possible values of $x$. [8 Marks]} 
    
    \item \textbf{Determine all values of $k$ for which the equation $$(2^x) + 2^{-x} = 3$$ has exactly one root. [8 Marks]}
    
    \item \textbf{Let $f(x) = 2^{kx} + 9$, where $k$ is a real number. If $$f(3):f(6) = 1:3$$, determine the value of $f(9) - f(3)$. [10 Marks]}
\end{enumerate}
\end{document}