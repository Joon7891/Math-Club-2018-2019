\documentclass[12pt]{article}
\usepackage[a4paper, total={7in, 10in}]{geometry}
\usepackage{amsmath}
\usepackage{amssymb}
\usepackage{graphicx}

\title{Week 1 Problem Set Solutions\vspace{-3mm}}
\author{2018-2019 SLSS Math Club\vspace{-5mm}}
\date{October 16, 2018 \vspace{-5mm}}

\begin{document}
\maketitle

\section*{Intermediate Problems}
\begin{enumerate}
    \item \textbf{Determine all values of $x$ such that $x - \frac{4}{x} = -3$. Do not use guess and check.} 
    \begin{align*}
            x-\frac{4}{x} &= -3 \\
            x^2-4 &= -3x \\
            x^2+3x-4 &= 0 \\
            (x+4)(x-1) &= 0 \\
    \end{align*}
    $\therefore x = -4 \text{ or } x = 1$
    
    \item \textbf{The points $A(5, -8), B(9, -30),$ and $C(n, n)$ lie on the same line. Determine the value of $n$.}
    \begin{align*}
        m &= \frac{-8-(-30)}{5-9} \\
        m &= -5.5 \\ \\
        y &= mx+b \\
        -8 &= (-5.5)(5)+b \\
        b &= 19.5 \\ \\
        y &= -5.5x+19.5 \\
        n &= -5.5n+19.5 \\
        6.5n &= 19.5 \\
        n &= 3
    \end{align*}
    $\therefore n = 3$
    
    \item \textbf{Given $x + y = 4$ and $xy = 12$, determine the value of $x^2 + 5xy + y^2$.} \\
    
    Carefully rearranging and factoring yields:    
    \begin{align*}
        &x^2+5xy+y^2 \\
        =& (x+y)^2+3xy \\
        =& (4)^2+3(12) \\
        =& 52
    \end{align*}
    
    $\therefore x^2 + 5xy + y^2 = 52$
    
\end{enumerate}
\newpage

\section*{Advanced Problems}
\begin{enumerate}
    \item \textbf{The functions $f$ and $g$ satisfy the following:
    \begin{align*}
        f(x) + g(x) & = 3x + 5 \\
        f(x) - g(x) &= 5x + 7
    \end{align*} for all values of $x$. Determine the value of $4f(2)g(2)$.} \\
    \begin{align*}
        f(x) + g(x) & = 3x + 5 \\
        + f(x) - g(x) &= 5x + 7 \\[-10pt]
        \cline{1-2}
        2f(x) &= 8x+12 \\
        f(x) &= 4x+6 \\ \\
        g(x) &= 3x+5-f(x) \\
        g(x) &= 3x+5-(4x+6) \\
        g(x) &= 3x+5-4x-6 \\
        g(x) &= -x-1 \\ \\
        f(2) &= 4(2)+6 \\
        f(2) &= 14 \\ \\
        g(2) &= -2-1 \\
        g(2) &= -3 \\ \\
        4f(2)g(2) &= 4(14)(-3) \\
        4f(2)g(2) &= -168
    \end{align*}
    
    $\therefore 4f(2)g(2) = -168$ 
    
    \item \textbf{If $a(a - 1) = b(b - 1)$, what is $a$ in terms of $b$?} \\
    
    Note: There was a small mistake in the problem statement, which may have caused some confusion. The statement should have included that $a \neq b$.
        \begin{align*}
        a(a-1) &= b(b-1) \\
        a^2-a &= b^2-b \\
        a^2-b^2 &= a-b \\
        (a-b)(a+b) &= a-b \\
    \end{align*}
    
    Since $a \neq b$, we can divide both sides by $a-b$:
    \begin{align*}
        a+b &= 1 \\
        a & = 1 - b
    \end{align*}
    
    $\therefore a = 1-b$

    \item \textbf{If $T = x^2 + \frac{1}{x^2}$, determine $a$ and $b$ such that $x^6 + \frac{1}{x^6} = T^3 + aT + b$.}
    \begin{align*}
        T &= x^2+\frac{1}{x^2} \\
        T^3 &= x^6+3x^2+\frac{3}{x^2}+\frac{1}{x^6} \\
        T^3-3x^2-\frac{3}{x^2} &= x^6+\frac{1}{x^6} \\
        T^3-3(x^2-\frac{1}{x^2}) &= x^6+\frac{1}{x^6} \\
        T^3-3T &= x^6+\frac{1}{x^6} \\
        T^3-3T &= T^3+aT+b
    \end{align*}
    
    $\therefore a = -3, b = 0$ 
    
\end{enumerate}
\end{document}
