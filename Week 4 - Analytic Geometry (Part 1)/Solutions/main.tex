\documentclass[12pt]{article}
\usepackage[a4paper, total={7in, 10in}]{geometry}
\usepackage{amsmath}
\usepackage{amssymb}
\usepackage{graphicx}

\title{Week 4 Problem Set: Analytic Geometry (Part 1)\vspace{-3mm}}
\author{2018-2019 SLSS Math Club\vspace{-5mm}}
\date{October 31, 2018 \vspace{-5mm}}

\begin{document}
\maketitle

\section*{Basic Problems}
\begin{enumerate}
    \item \textbf{Consider $\Delta ABC$ where $A(4, 3)$, $B(-6, -1)$, and $C(2, -5)$. If $D$ is the midpoint of $\overline{AB}$ and $E$ is the midpoint of $\overline{AC}$, prove that $\overline{DE}$ is half the length of $\overline{BC}$}
    \begin{align*}
        D&=(\frac{4-6}{2}, \frac{3-1}{2}) \\
        D&=(-1, 1) \\ \\
        E&=(\frac{4+2}{2}, \frac{3-5}{2}) \\
        E&=(3, -1) \\ \\
        d_{DE}&=\sqrt{(3+1)^2+(-1-1)^2} \\
        d_{DE}&=\sqrt{20} \\ \\
        d_{BC}&=\sqrt{(2+6)^2+(-5+1)^2} \\
        d_{BC}&=\sqrt{80} \\ 
        d_{BC}&=\sqrt{(4)(20)} \\
        d_{BC}&=2\sqrt{20} \\ 
    \end{align*}
    $\because 2d_{DE} = d_{BC}, Q.E.D$
    
    \item \textbf{The line $2x + y = 0$ is rotated $90^{\circ}$ about the origin. Determine the equation of the new line.} \\ \\
    Given that the line is rotated $90^{\circ}$ about the origin, the new line will be perpendicular to the original line. This means that the new slope will be the negative reciprocal of the original slope. \\ \\
    As well, the new line will also have a $y$-intercept of $0$. \\ \\
    Therefore, the equation of the new line is:
    \begin{equation*}
        y = -\frac{1}{2}x
    \end{equation*}
    
    \end{enumerate} \newpage
\section*{Intermediate Problems}
\begin{enumerate}
    \item \textbf{Find the coordinates of all points in the Cartesian plane that are equidistant from the $x$-axis, the $y$-axis and the point $(2, 1)$} \\ \\
    Since the points are equidistant from the $x$-axis and the $y$-axis, we know that the points will lie on the line $y=x$. The distances between the line and the point $(2, 1)$ can be expressed in the following function:
    \begin{align*}
        y &= \sqrt{(2 - x)^2 + (1 - x)^2} \\
        &= \sqrt{2x^2 - 6x + 5}
    \end{align*}
    
    We will now find the intersections between the $2$ functions:
    \begin{align*}
        y &= x \\
        y &= \sqrt{2x^2 - 6x + 5} \\
        x &= \sqrt{2x^2 - 6x + 5} \\
        x^2 & = 2x^2 - 6x + 5 \\
        0 &= x^2 - 6x + 5 \\
        0 &= (x - 1)(x - 5) \\
        x &= 1 \text{ or } x = 5
    \end{align*}    

    $\therefore$ The solutions are $(1, 1)$ and $(5, 5)$.
    
    \item \textbf{$\Delta ABC$ is an isosceles triangle with vertex $A$ at $(0, 0)$, and vertices $B$ and $C$ on the line $2x+3y-13 = 0$ and $\angle BAC = 90^{\circ}$. Determine the coordinates of $B$ and $C$} \\ \\
    Since $\Delta ABC$ is an isosceles triangle and $\angle BAC = 90^{\circ}$, we know that $AB = AC$. \\ \\
    Let the coordinates of $B$ be $(h, k)$. Since $C$ is the image of $B$ after a rotation of $90^{\circ}$ about the origin, the coordinates of $C$ are $(-k, h)$. We can now substitute these coordinates in the equation of the line to solve for $k$ and $h$.
    \begin{align*}
        2h+3k-13&=0 \\
        -2k+3h-13&=0 \\ \\
        4h+6k&=26 \\
        9h-6k&=39 \\[-11pt]
        \cline{1-2}
        13h&=65 \\
        h&=5 \\
        k&=1
    \end{align*}
    $\therefore$ The coordinates are $B(5, 1)$ and $C(-1, 5)$ \newpage
    
\end{enumerate}

\section*{Advanced Problem}
\begin{enumerate}
    \item \textbf{Two straight lines are defined by the equation $6x^2 + xy - 2y^2 + 11x + 5y + 3 = 0$. Determine the product of the slopes of the two lines.} \\ \\
    This is an application of zero product rule - where if $ab = 0$, either $a = 0$ or $b = 0$. Therefore, in order to create an equation that defines two lines, let $a$ and $b$ represent the standard form of the two lines. \\ \\
    The slope of a line in standard form, $Ax + By + C = 0$ is $-\frac{A}{B}$. Given that the above equation represents the products of the two lines, the products of their slopes will be the coefficient of the $x^2$ term divided by the coefficient of the $y^2$ term, giving us the following:
    \begin{align*}
        m_1m_2 &= \frac{6}{(-2)} \\
        &= -3
    \end{align*}

    $\therefore$ The product of the slopes of two lines is $-3$.
\end{enumerate}

\end{document}