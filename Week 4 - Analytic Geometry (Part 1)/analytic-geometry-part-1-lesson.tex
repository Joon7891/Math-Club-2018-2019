\documentclass[12pt]{article}
\usepackage[a4paper, total={7in, 10in}]{geometry}
\usepackage{amsmath}
\usepackage{graphicx}

\title{Analytic Geometry (Part 1) Lesson\vspace{-3mm}}
\author{2018-2019 SLSS Math Club\vspace{-5mm}}
\date{October 31, 2018\vspace{-5mm}}

\begin{document}
\maketitle
\section{Introduction}
Analytic geometry, also known as coordinate geometry or Cartesian geometry, is the study of geometry using a coordinate system.

\section{Formulas}
\subsection{Distance Formula}
The distance between two points $(x_1, y_1)$ and $(x_2, y_2)$ is given by the following formula:
\begin{equation*}
    d = \sqrt{(x_2 - x_1)^2 + (y_2 - y_1)^2}
\end{equation*}

\subsection{Midpoint Formula}
The midpoint between two points $(x_1, y_1)$ and $(x_2, y_2)$ is given by the following formula:
\begin{equation*}
    M = (\frac{x_1 + x_2}{2}, \frac{y_1 + y_2}{2})
\end{equation*}

\subsection{Shortest/Perpendicular Distance}
The shortest distance from a point $(x_1, y_1)$ to a line $Ax + By + C = 0$ is given by:
\begin{equation*}
    d = \frac{|Ax_1 + By_1 + C|}{\sqrt{A^2 + B^2}}
\end{equation*}

\section{Triangle Properties}
\subsection{Median}
A line segment joining a vertex to the midpoint of the opposing side, bisecting it. \\ \\
An linear equation of a median can be found by determining the midpoint of the opposite side and using the midpoint and the vertex as points to determine its $y = mx + b$ equation.

\subsection{Altitude}
A line segment that goes through a vertex and is perpendicular to the opposide side. \\ \\
A linear equation of an altitude can be found by determining the slope of the opposite side and using its perpendicular slope along with the vertex to determine its $y = mx + b$ equation.

\subsection{Right Bisector}
A line that goes through a midpoint at a perpendicular angle. \\ \\
A linear equation of a right bisector can be found by determining the slope and the midpoint of a given side and using its perpendicular slope and its midpoint to determine its $y = mx + b$ equation.    

\newcommand{\spacing}{\\ \\ \\ \\ \\ \\}
\section{Practice}
\begin{enumerate}
    \item For $\Delta ABC$, for which its verticies are $A(1, 2)$, $B(4, 5)$, and $C(-2, 7)$, determine the following:
    \begin{enumerate}
        \item The perimeter of $\Delta ABC$ \spacing
        \item The linear equation of the median from $A$ to side $\overline{BC}$ \spacing
        \item The linear equation of the altitude from $B$ to side $\overline{AC}$ \spacing
        \item The linear equation of the right bisector from $C$ to side $\overline{AB}$ \spacing
    \end{enumerate}

    \item Determine the perpendicular distance between the point $A(7, 1)$ and the line going through points $B(5, -2)$ and $C(2, 14)$ \spacing
\end{enumerate}

\end{document}