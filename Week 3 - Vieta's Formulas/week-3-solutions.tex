\documentclass[12pt]{article}
\usepackage[a4paper, total={7in, 10in}]{geometry}
\usepackage{amsmath}
\usepackage{amssymb}
\usepackage{graphicx}

\title{Week 3 Problem Set Solutions\vspace{-3mm}}
\author{2018-2019 SLSS Math Club \vspace{-5mm}}
\date{October 30, 2018 \vspace{-5mm}}

\begin{document}
\maketitle 

\section*{Basic Problems}
\begin{enumerate}
    \item \textbf{The roots of $x^2 + px + q = 0$ are $5$ and $-2$. What are the values of $p$ and $q$?}
    \begin{align*}
        x_1+x_2&=\frac{-b}{a} \\
        x_1x_2&=\frac{c}{a}
    \end{align*}
    
    Substitute the given values into the two equations to solve for $p$ and $q$.
    \begin{align*}
        5-2&=-p \\
        3&=-p \\
        p&=-3 \\
        (5)(-2)&=q \\
        q&=-10
    \end{align*}
    
    $\therefore p = -3$ and $q = -10$
    
    \item \textbf{Let $m$ and $n$ be the two roots of the equation $x^2 - 15x + 28 = 0$. Find $(m + 1)(n + 1)$} \\
    
    First, we rearrange the expression by distributing the terms in order to be able to use Vieta's formulas.
    \begin{align*}
        &(m+1)(n+1) \\
        &=mn+m+n+1 \\
        &=mn+(m+n)+1 \\ \\
        m+n&=15 \\
        mn&=28 \\
    \end{align*}
    
    Substitute the values of $mn$ and $m+n$ back into the expression to find its value.
    \begin{align*}
        &=mn+(m+n)+1 \\
        &=28+15+1 \\
        &=44
    \end{align*}
    
    $\therefore (m+1)(n+1)=44$
    
\end{enumerate}
\section*{Intermediate Problems}
\begin{enumerate}
    \item \textbf{The roots for the equation $x^2 + 4x - 5 = 0$ are also the roots of the equation $2x^3 + 9x^2 - 6x - 5 = 0$. What is the third root of the second equation?} \\
    
    We begin by finding the sum of the roots for the first equation.
    \begin{align*}
        x_1+x_2&=-4
    \end{align*}
    
    We then find the sum of the roots for the second equation.
    \begin{align*}
        x_1+x_2+x_3=\frac{-9}{2}
    \end{align*}
    
    If we let $x_3$ be the third root of the second equation, then we can subtract the two sums to find the value of $x_3$.
    \begin{align*}
        x_3&=\frac{-9}{2}+4 \\
        x_3&=\frac{-1}{2}
    \end{align*}
    
    $\therefore$ The third root of the second equation is $\frac{-1}{2}$
    
    \item \textbf{Let $g(x) = x^3 - 5x^2 + 2x - 7$, and let the roots of $g(x)$ be $p, q,$ and $r$. Compute $p^2qr + pq^2r + pqr^2$} \\
    
    We begin by factoring out $pqr$ in order to be able to use Vietta's formulas.
    \begin{align*}
        p^2qr+pq^2r+pqr^2&=pqr(p+q+r)
    \end{align*}
    
    We can then very easily evaluate the expression
    \begin{align*}
        pqr&=7 \\
        p+q+r&=5 \\
        pqr(p+q+r)&=(5)(7)=35
    \end{align*}
    
    $\therefore p^2qr+pq^2r+pqr^2=35$
    
\end{enumerate}
\newpage

\section*{Advanced Problems}
\begin{enumerate}
    \item \textbf{Let $a$ and $b$ be the roots of the equation $x - mx + 2 = 0$. Suppose that $a + \frac{1}{b}$  and $b + \frac{1}{a}$ are the roots of the equation $x^2 - px + q = 0$. What is the $q$?} \\
    
    We begin by noticing that $q$ is equal to the product of the roots. This means that we can find $q$ by finding the product of the roots.
    \begin{align*}
        q&=(a+\frac{1}{b})(b+\frac{1}{a}) \\
        q&=ab+\frac{1}{ab}+2
    \end{align*}
    
    Afterwards, we notice that $ab$ is the product of the roots of the first equation which is also equal to 2. This information is all we need in order to find the value of $q$.
    \begin{align*}
        ab&=2 \\
        q&=2+\frac{1}{2}+2 \\
        q&=\frac{9}{2}
    \end{align*}
    
    $\therefore q=\frac{9}{2}=4.5$
    
    \item \textbf{If the sums of the reciprocals of the roots of the quadratic $3x^2 + 7x + k = 0$ is $\frac{7}{3}$, what is $k$?} \\
    
    Let $a$ and $b$ represent the roots of the quadratic.
    \begin{align*}
        \frac{1}{a}+\frac{1}{b}&=\frac{7}{3} \\
        \frac{a+b}{ab}&=\frac{7}{3} \\
        ab&=\frac{3(a+b)}{7}
    \end{align*}
    
    Since $a+b$ equals $\frac{-7}{3}$, we can substitute it into the equation to solve for $ab$ which is equal to $k$.
    \begin{align*}
        ab&=\frac{3(\frac{-7}{3})}{7} \\
        ab&=\frac{-7}{7} \\
        ab&=-1 \\
        k&=-1
    \end{align*}
    
    $\therefore k=-1$
    
\end{enumerate}
\end{document}