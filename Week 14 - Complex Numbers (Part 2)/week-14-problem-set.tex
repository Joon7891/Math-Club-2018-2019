\documentclass[12pt]{article}
\usepackage[a4paper, total={7in, 10in}]{geometry}
\usepackage{amsmath}
\usepackage{amssymb}
\usepackage{graphicx}

\title{Week 14 Problem Set: Complex Numbers (Part 2)\vspace{-3mm}}
\author{2018-2019 SLSS Math Club\vspace{-5mm}}
\date{May 8, 2019\vspace{-5mm}}

\newcommand{\bspace}{\\ \\ \\ \\} 
\newcommand{\ispace}{\\ \\ \\ \\ \\ \\ \\}
\newcommand{\aspace}{\\ \\ \\ \\ \\ \\ \\ \\ \\ \\ \\ \\ \\}

\begin{document}
\maketitle

\section*{Basic Problems}
\begin{enumerate}
    \item Evaluate $e^{i\pi}$. \bspace
    \item Evaluate $5^{4i + 7}$. \bspace
    \item Determine the sum of all $1729^{\text{th}}$ roots of unity. \bspace
\end{enumerate}

\section*{Intermediate Problems}
\begin{enumerate}
    \item Solve $\cos{x} = 2$ for $x \in \mathbb{C}$. \ispace
    \item Determine the resulting point when the point $(-1, 3)$ is rotated $\frac{\pi}{8}$ radians counter clockwise about the origin. \ispace
    \item If 1, $\delta_1$, $\delta_2$, $\delta_3$, are distinct fourth roots of unity, then evaluate the expression below.
    \begin{equation*}
        \frac{31 - 2\delta_1}{1 - 2\delta_1} + \frac{31 - 2\delta_2}{1 - 2\delta_2} +\frac{31 - 2\delta_3}{1 - 2\delta_3}
    \end{equation*}  \ispace
\end{enumerate}

\section*{Advanced Problems}
\begin{enumerate}
    \item Using De Moivre's theorem, solve for $x \in \mathbb{C}$
    \begin{equation*}
        0 = x^3 + \frac{1}{x^3} + x^2 + \frac{1}{x^2} + x + \frac{1}{x}
    \end{equation*} \aspace
    
    \item Evaluate $$\sum_{n = 0}^{\infty} \frac{\cos{(n\theta)}}{2^n}$$ where $\cos{\theta} = \frac{1}{5}$. \aspace
\end{enumerate}
\end{document}