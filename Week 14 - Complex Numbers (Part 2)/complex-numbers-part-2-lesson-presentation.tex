\documentclass{beamer}
\usepackage[utf8]{inputenc}
\usetheme{Antibes}
\usecolortheme{beaver}
\setbeamertemplate{navigation symbols}{}

\title{Complex Numbers (Part 2) Lesson}
\author{2018-2019 SLSS Math Club}
\date{May 8, 2019}

\begin{document}
\frame{\titlepage}

\section{Euler's Formula}
\begin{frame}{Euler's Formula}
\begin{itemize}
    \item In complex analysis, Euler's provides a fundamental bridge between exponential and trigonometric functions. 
    \item It can be used to evaluate complex exponents into a complex number.
\end{itemize}

\begin{equation*}
    e^{ix} = \text{cis } x = \cos{x} + i\sin{x}
\end{equation*}
\end{frame}

\begin{frame}{Geometric Interpretation}
Euler's formula allows for any complex number to be represented in the form $e^{ix}$ which lies on the imaginary unit circle and has real and imaginary components $\cos{x}$ and $\sin{x}$, respectively.

\begin{center}
    \includegraphics[scale = 0.3]{"Week 14 - Complex Numbers (Part 2)/Graphics/Euler's Formula".png}
\end{center}
\end{frame}

\begin{frame}{Resolving Complex Numbers onto the Unit Circle}
Using the following formulas, we can represent any complex number $z = a + ib$ as $z = r(\cos{x} + i\sin{x})$.
\begin{align*}
    r = \sqrt{a^2 + b^2} && \cos{\theta} = \frac{a}{r} && \sin{\theta} = \frac{b}{r}
\end{align*}
\end{frame}

\begin{frame}{Trigonometric Interpretation}
Using $e^{ix} =  \cos{x} + i\sin{x}$ and $e^{-ix} =  \cos{x} - i\sin{x}$, we can extend the definition of trigonometric function into the complex domain.
\begin{align*}
    \cos{x} = \frac{e^{ix} +e^{-ix}}{2} && \sin{x} = \frac{e^{ix} - e^{-ix}}{2i} && \tan{x} = \frac{e^{ix} - e^{-ix}}{i(e^{ix} +e^{-ix})}
\end{align*}
\end{frame}

\section{De Moivre's Theorem}
\begin{frame}{De Moivre's Theorem}
    De Moivre’s theorem gives a formula for computing powers of complex numbers.
    \begin{equation*}
        (r(\cos{\theta} + i\sin{\theta}))^n = r^n(\cos{(n\theta) + i\sin(n\theta)}
    \end{equation*}
\end{frame}
\end{document}