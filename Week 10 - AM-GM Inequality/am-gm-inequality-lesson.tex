\documentclass[12pt]{article}
\usepackage[a4paper, total={7in, 10in}]{geometry}
\usepackage{amsmath}
\usepackage{graphicx}

\title{AM-GM Inequality Lesson\vspace{-3mm}}
\author{2018-2019 SLSS Math Club\vspace{-5mm}}
\date{February 6, 2018\vspace{-5mm}}

\begin{document}
\maketitle

\section{Definitions}
\subsection{Arithmetic Mean}
A type of mean, calculated as the sum of $n$ numbers divided by $n$.

\begin{equation*}
    \text{AM} = \frac{a_{1} + a_{2} + \dots + a_{n - 1} + a_{n}}{n}
\end{equation*}

\subsection{Geometric Mean}
A type of mean, calculated as the $n$-th root of the product of $n$ numbers.

\begin{equation*}
    \text{GM} = \sqrt[n]{a_{1} a_{2} \dots a_{n - 1} a_{n}}
\end{equation*}

\subsection{Arithmetic Mean - Geometric Mean Inequality}
States that the arithmetic mean of a given set is equal to or greater than the geometric mean of the same set if all elements are non-negative ($a_i \geq 0$)

\begin{align*}
    \text{AM} & \geq \text{GM} \\
     \frac{a_{1} + a_{2} + \dots + a_{n - 1} + a_{n}}{n} & \geq \sqrt[n]{a_{1} a_{2} \dots a_{n - 1} a_{n}}
\end{align*}

\subsubsection{Equality Case}
Arithmetic and geometric means are equal if and only if all elements are identical.

\begin{equation*}
    a_{1} = a_{2} = \dots = a_{n - 1} = a_{n}
\end{equation*}

\section{Applications}
\subsection{Proving Inequalities}

\subsection{Simplifying Equations}
\subsection{Determining Minimum Value}
\subsection{Determining Maximum Value}

\end{document}