\documentclass[12pt]{article}
\usepackage[a4paper, total={7in, 10in}]{geometry}
\usepackage{amsmath}
\usepackage{graphicx}
\usepackage{setspace}
\setstretch{1.25}
\newcommand{\tdiv}{\; | \;}

\title{Week 11 Problem Set: Number Theory (Part 1)\vspace{-3mm}}
\author{2018-2019 SLSS Math Club\vspace{-5mm}}
\date{February 27, 2019\vspace{-5mm}}

\begin{document}
\maketitle

\section*{Basic Problems}
\begin{enumerate}
    \item \textbf{Determine $(8 \cdot 16) \pmod{7}$}
    
    Applying our modulus rules, we achieve the following
    \begin{align*}
        8 &\equiv 1 \pmod{7} \\
        16 &\equiv 2 \pmod{7} \\
        8 \cdot 16 &\equiv 2 \pmod{7}
    \end{align*}
    
    Therefore, $(8 \cdot 16) \pmod{7} = 2$
    
    \item \textbf{Determine the remainder when $124 \cdot 134 \cdot 23 \cdot 49 \cdot 235 \cdot 13$ is divided by $3$.}
    
    Determine the remainder when the above product is divided by $3$ is equivalent to determine an integer $n < 3$ which is congruent to the product modulo $3$.
    \begin{align*}
        124 \cdot 134 \cdot 23 \cdot 49 \cdot 235 \cdot 13 &\equiv 1 \cdot 2 \cdot 2 \cdot 1 \cdot 1 \cdot 1 \pmod{3} \\
        4 &\equiv 1 \pmod{3}
    \end{align*}
    
    Therefore, the remainder is $1$.
    
    \item \textbf{Determine the remainder when $124 + 234 + 43 + 56 + 23 + 12 + 78$ is divided by $3$.}
    
    Determining the remainder when the above sum is divided by $3$ is equivalent to determining an integer $n < 3$ which is congruent to the sum modulo $3$.
    \begin{align*}
        124 + 234 + 43 + 56 + 23 + 12 + 78 &\equiv 1 + 0 + 1 + 2 + 2 + 0 + 0 \pmod{3} \\
        6 &\equiv 0 \pmod 3
    \end{align*}
    
    Therefore, the remainder is $0$.
    
    \item \textbf{Show that if $a - b$ and $a + b$ is divisible by $n$, that $a^2 - b^2$ is divisible by $n$.}
    
    Since $n$ divides $a + b$ and $a - b$, from our division properties, we can state that $n$ divides $a^2 - b^2$ since $a^2 - b^2$ can be rewritten as $(a - b)(a + b)$.
    
\end{enumerate}

\section*{Intermediate Problems}
\begin{enumerate}
    \item \textbf{Show that a square is always in the $4k$ or $4k + 1$.} 
    
    Since all integers are either even or odd, any integer $n$ can be written as $2m$ or $2m + 1$, for some integer $m$. As such, all perfect squares can be written as:
    \begin{align*}
        n &= 2m && n &= 2m + 1 \\
        n^2 &= 4m^2 && n^2 &= 4m^2 + 4m + 1 \\
    \end{align*}
    
    
    
    \item \textbf{Determine all integers $a$ such that $\displaystyle{\frac{5}{a + 2}}$ is an integer.} 
    
    In order for $\displaystyle{\frac{5}{a + 2}}$ to be an integer, $a + 2$ must be a factor of $5$. The factors are $\pm 1$ and $\pm 5$. \vspace{-0.5em}
    
    \begin{align*}
        a + 2 = +1 && a + 2 = -1 && a + 2 = +5 && a + 2 = -5 \\
        a = -1 && a = -3 && a = +3 && a = -7
    \end{align*}
    
    Therefore, $a = 3, -1, -3$ and $-7$.
    
    \item \textbf{Suppose that $n$ is a positive integer and $a = \displaystyle{\frac{10^{2n} - 1}{3(10^n + 1)}}$. If the sum of the digits of $a$ is $567$, determine the value of $n$.} 
    
    Factoring the numerator and simplifying, we are left with the following equation:
    \begin{align*}
        a &= \frac{10^{2n} - 1}{3(10^n + 1)} \\
        &= \frac{(10^n - 1)(10^n + 1)}{3(10^n + 1)} \\
        &= \frac{10^n - 1}{3}
    \end{align*}
    
    From observation, we recognize that the equation $\displaystyle{a = \frac{10^n - 1}{3}}$ is an integer consisting of $n$ $3$s. As such, the sum of the digits of $a$ is $3n$, which we can use to solve for $n$.
    \begin{align*}
        3n &= 567 \\
        n &= 189
    \end{align*}
    
    Therefore, $n = 189$.
\end{enumerate}

\section*{Advanced Problems}
\begin{enumerate}
    \item \textbf{Show that if $p > 3$ is prime, that $p^2 \equiv 1 \pmod{24}$}
    
    Rewriting the above statement, we can state the following:
    \begin{align*}
        p^2 \equiv 1 \pmod{24} \\
        p^2 - 1 \equiv 0 \pmod{24} \\
        (p - 1)(p + 1) \equiv 0 \pmod{24}
    \end{align*}
    
    Since all primes greater than $3$ are odd, $p$ must be congruent to either $1, 3, 5$ or $7$ modulo $8$. In all these cases, $p^2 - 1$ is divisible by $8$.
    \begin{align*}
        p \equiv 1 \pmod{8} && p \equiv 3 \pmod{8} && p \equiv 5 \pmod{8} && p \equiv 7 \pmod{8} \\
        p^2 \equiv 1 \pmod{8} && p^2 \equiv 1 \pmod{8} && p^2 \equiv 1 \pmod{8} && p^2 \equiv 1 \pmod{8} \\
        p^2 - 1 \equiv 0 \pmod{8} && p^2 - 1 \equiv 0 \pmod{8} && p^2 - 1 \equiv 0 \pmod{8} && p^2 - 1 \equiv 0 \pmod{8} 
    \end{align*}
    
    Since $p$ is prime, $p$ cannot be divisible by $3$. As such, $p \equiv 1 \pmod{3}$ or $p \equiv 2 \pmod{3}$, implying that either $p - 1$ or $p + 1$ is divisible by $3$, and thus $p^2 - 1$ is divisible by $3$.
    
    Since $p^2 - 1$ is divisible by $3$ and $8$, we can state that $p^2 - 1$ is divisible by $24$, as desired.
    
    \item \textbf{Show that $n^2 + 23$ is divisible by $24$ for infinitely many $n$.}
    
    Consider $n = 24k + 1$ for some integer $k$. This would yield the following:
    \begin{align*}
        n^2 + 23 &= (24k + 1)^2 + 23 \\
                &= 24^2k^2 + 2\cdot 24k + 1 + 23 \\
                &= 24^2k^2 + 2\cdot 24k + 24 \\
                &= 24(24k^2 + 2k + 1)
    \end{align*}
    
    Since $24 \tdiv 24$, we can state that $24 \tdiv 24(24k^2 + 2k + 1)$. Since there exists infinitely many $n$ in the form $n = 24k + 1$, there exists infinitely many $n$ such that $n^2 + 23$ is divisible by $24$.
    
\end{enumerate}

\end{document}