\documentclass{beamer}
\usepackage[utf8]{inputenc}
\usetheme{Antibes}
\usecolortheme{beaver}
\setbeamertemplate{navigation symbols}{}
\newcommand{\ndiv}{\hspace{-4pt}\not|\hspace{2pt}}

\title{Number Theory (Part 1) Lesson}
\author{2018-2019 SLSS Math Club}
\date{February 27, 2019}

\begin{document}
\frame{\titlepage}

\section{Introduction}
\begin{frame}{Introduction}
     Number theory is a branch of mathematics that studies integers. \newline

    Some of the sub-branches of number theory are:
    \begin{itemize}
        \item Number bases
        \item Divisibility
        \item Factorization 
        \item Prime numbers 
        \item Modular arithmetic
        \item Diophantine equations
        \item Greatest common divisor, lowest common multiple
    \end{itemize}
\end{frame}

\section{Fundamental Theorem of Arithmetic}
\begin{frame}{Fundamental Theorem of Arithmetic}
    
    The fundamental theorem of arithmetic is a theorem that states that every integer greater than 1 is either prime, or the product of a unique combination of primes. \newline
    
    That is, for every n ≥ 2, it can be represented as $$n = p_1^{\alpha_1}p_2^{\alpha_2}\dots p_i^{\alpha_i}$$ wher $p$ is prime. \newline
    
    \textbf{Prime:} A number that is only divisible by one and itself
\end{frame}

\begin{frame}{Practice}
    Represent the following integers as a product of primes: \newline
    
    \textbf{Example:} $24 = 2^3 \cdot 3^1$
    \begin{align*}
        7 && 270 && 2018
    \end{align*}
    
    \vspace{8em}
\end{frame}

\section{Divisibility}
\begin{frame}{Divisibility}
    A divisor of $n$, also called a factor of $n$, is an integer $m \neq 0$, such that $m$ divides $n$ to produce an integer ($\frac{n}{m}$). \newline
    
    This is written is $m | n$. Conversely, if $m$ does not divide $n$, it is written is $m \ndiv n$. \newline
    
    Equivalently, this also means that there exists an integer $a$, such that $n = am$. \newline
    
    \textbf{Examples:}
    \begin{align*}
        
    \end{align*}
    
\end{frame}

\begin{frame}{Properties}
    
\end{frame}

\begin{frame}{Practice}
    
\end{frame}

\section{Modular Arithmetic}
\begin{frame}{Modular Arithmetic}
A divisor of $n$, also called a factor of $n$, is an integer $m \neq 0$, such that $m$ divides $n$ to produce an integer ($\frac{n}{m}$). This is written is $m | n$. Conversely, if $m$ does not divide $n$, it is written is $m \ndiv n$. Equivalently, this also means that there exists an integer $a$, such that $n = a \ndiv m$.
\end{frame}

\begin{frame}{Modulus}
    
\end{frame}

\begin{frame}{Congruence}
    
\end{frame}

\begin{frame}{Properties}
    
\end{frame}

\end{document}