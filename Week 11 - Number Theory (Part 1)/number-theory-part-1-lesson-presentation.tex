\documentclass{beamer}
\usepackage[utf8]{inputenc}
\usetheme{Antibes}
\usecolortheme{beaver}
\setbeamertemplate{navigation symbols}{}
\newcommand{\ndiv}{\hspace{-4pt}\not|\hspace{2pt}}
\newcommand{\tdiv}{\; | \;}

\title{Number Theory (Part 1) Lesson}
\author{2018-2019 SLSS Math Club}
\date{February 27, 2019}

\begin{document}
\frame{\titlepage}

\section{Introduction}
\begin{frame}{Introduction}
     Number theory is a branch of mathematics that studies integers. \newline

    Some of the sub-branches of number theory are:
    \begin{itemize}
        \item Number bases
        \item Divisibility
        \item Factorization 
        \item Prime numbers 
        \item Modular arithmetic
        \item Diophantine equations
        \item Greatest common divisor, lowest common multiple
    \end{itemize}
\end{frame}

\section{Fundamental Theorem of Arithmetic}
\begin{frame}{Fundamental Theorem of Arithmetic}
    
    The fundamental theorem of arithmetic is a theorem that states that every integer greater than 1 is either prime, or the product of a unique combination of primes. \newline
    
    That is, for every n ≥ 2, it can be represented as $$n = p_1^{\alpha_1}p_2^{\alpha_2}\dots p_i^{\alpha_i}$$ where $p$ is prime. \newline
    
    \textbf{Prime:} A number that is only divisible by one and itself
\end{frame}

\begin{frame}{Practice}
    Represent the following integers as a product of primes: \newline
    
    \textbf{Example:} $24 = 2^3 \cdot 3^1$
    \begin{align*}
        7 && 270 && 2018
    \end{align*}
    
    \vspace{8em}
\end{frame}

\section{Divisibility}
\begin{frame}{Divisibility}
    A divisor of $n$, also called a factor of $n$, is an integer $m \neq 0$, such that $m$ divides $n$ to produce an integer ($\frac{n}{m}$). \newline
    
    This is written is $m | n$. Conversely, if $m$ does not divide $n$, it is written is $m \ndiv n$. \newline
    
    Equivalently, this also means that there exists an integer $a$, such that $n = am$. \newline
    
    \textbf{Examples:}
    \begin{align*}
        2\tdiv 10 && 6\; \ndiv \; 15 && 7 \tdiv 42
    \end{align*}
\end{frame}

\begin{frame}{Properties}
    For all integers $a, b, c$, the following properties hold: 
    \begin{itemize}
        \item $1 \tdiv a, -1 \tdiv a, a \tdiv 0,$ and $a \tdiv a$
        \item If $a \tdiv b$ and $b \tdiv c$, then $a \tdiv c$
        \item If $a \tdiv b$ and $b  \tdiv a$, then $a = b$ or $a = -b$
        \item If $a \tdiv b$ and $a \tdiv c$, then $a  \tdiv (n \cdot b + m \cdot c)$ for all integers $n$ and $m$
    \end{itemize}
\end{frame}

\begin{frame}{Practice}
    \vspace{-5em}
    If $17 \tdiv (2a + 3b)$, prove that $17 \tdiv (9a + 5b)$
    
    \vspace{6em}
    
    If $7 \tdiv (3x + 2)$, prove that $7 \tdiv (15x^2 -11x - 14)$
\end{frame}

\section{Modular Arithmetic}
\begin{frame}{Modular Arithmetic}
    Modular arithmetic is a system of arithmetic for integers, which considers the remainder. \newline
    
    In modular arithmetic, numbers ``wrap-around'' a given integer to leave a remainder. \newline
    
    This branch of mathematics appears in computer science, computer algebra, and cryptography.
\end{frame}

\begin{frame}{Modulus}
    Given two positive integers $a$ and $b$, $a$ \textit{modulo} $b$ ($a \bmod b$), is the remainder when $a$ is divided by $b$. \newline

    \textbf{Examples:}
    \begin{align*}
        10 \bmod 5 = 0 && 32 \bmod 6 = 2 && 61 \bmod 12 = 1
    \end{align*}
\end{frame}

\begin{frame}{Congruence}
    For a positive integer $n$, integers $a$ and $b$ are said to be \textit{congruent} modulo $n$, if their remainders when divided by $n$ are the same. $$a \equiv b \pmod n$$
    
    \textbf{Examples:}
    \begin{align*}
        12 \equiv 0 \pmod 2 && 9 \equiv 3 \pmod 6 && 20 \equiv 10 \pmod 5
    \end{align*}
    
    \textbf{Practice:} State whether the following are true:
    \begin{align*}
        64 \equiv 36 \pmod 4 && 121 \equiv 13 \pmod {10} && 52 \equiv 2 \pmod 5
    \end{align*}
\end{frame}

\begin{frame}{Addition Properties}
    \begin{itemize}
        \item If $a + b = c$, then $a \bmod N + b \bmod N \equiv c \bmod N$
        \item If $a \equiv b \pmod N$, then $a + k \equiv b + k \pmod N$, for all integers $k$
        \item If $a \equiv b \pmod N$ and $c \equiv d \pmod N$, then $a + c \equiv b + d \pmod N$
        \item If $a \equiv b \pmod N$, then $-a \equiv -b \pmod N$
    \end{itemize}
\end{frame}

\begin{frame}{Multiplication and Exponentiation Properties}
    \begin{itemize}
        \item If $a \cdot b = c$, then $a \bmod N \cdot b \bmod N \equiv c \bmod N$
        \item If $a \equiv b \pmod N$, then $ka \equiv kb \pmod N$, for all integers $k$
        \item If $a \equiv b \pmod N$ and $c \equiv d \pmod N$, then $ac \equiv bd \pmod N$
        \item If $a \equiv b \pmod N$, then $a^k \equiv b^k \pmod N$ for all integers $k$
    \end{itemize}
\end{frame}

\begin{frame}{Practice}
    \vspace{-2em}
    Determine the last two digits of $11^{32}$.     \vspace{4em}
    
    Determine the remainder when $2^{800}$ is divided by $3$ \vspace{4em}

    
    If $a \equiv b \pmod N$, show that $8a(a^{20} + a^4) \equiv 8b(b^{20} + b^4)$ \vspace{4em}
    
\end{frame}
\end{document}