\documentclass[12pt]{article}
\usepackage[a4paper, total={7in, 10in}]{geometry}
\usepackage{amsmath}
\usepackage{graphicx}
\usepackage{multicol}
\usepackage{tabularx}
\setlength{\parindent}{0pt}
\newcommand{\ndiv}{\hspace{-4pt}\not|\hspace{2pt}}
\newcommand{\tdiv}{\; | \;}

\title{Number Theory (Part 1) Lesson\vspace{-3mm}}
\author{2018-2019 SLSS Math Club\vspace{-5mm}}
\date{February 27, 2019\vspace{-5mm}}

\begin{document}
\maketitle

\section{Introduction}
Number theory is a branch of mathematics devoted to the study of integers. Some of the sub-branches of number theory are as follows: divisibility, prime numbers, factorization, greatest common divisor, lowest common multiple, number bases, modular arithmetic, Diophantine equations.

\section{Fundamental Theorem of Arithmetic}
The fundamental theorem of arithmetic is a theorem that states that every integer greater than $1$ is either prime, or the product of a unique combination of primes. That is, for every $n \geq 2$, it can be expressed as $$n = p_1^{\alpha_1}p_2^{\alpha_2}\dots p_i^{\alpha_i}$$ where $p$ is a prime. \\

\textbf{Prime:} A number that is only divisible by one and itself. \\

\textbf{Examples:}
\begin{align*}
    7 = 7^1  && 24 = 2^3 \cdot 3^1 && 2016 = 2^5 \cdot 3^2 \cdot 7^1 
\end{align*}

\section{Divisibility}
A divisor of $n$, also called a factor of $n$, is an integer $m \neq 0$, such that $m$ divides $n$ to produce an integer ($\frac{n}{m}$). This is written is $m | n$. Conversely, if $m$ does not divide $n$, it is written is $m \ndiv n$. Equivalently, this also means that there exists an integer $a$, such that $n = am$. \\

\textbf{Examples:}
\begin{align*}
    2 \tdiv 10 && 6 \ndiv 15 && 7  \tdiv 42
\end{align*}

\subsection{Properties}
For all integers $a, b, c$, the following properties hold: 
\begin{itemize}
    \item $1  \tdiv a, -1  \tdiv a, a  \tdiv 0,$ and $a  \tdiv a$
    \item If $a  \tdiv b$ and $b  \tdiv c$, then $a  \tdiv c$
    \item If $a  \tdiv b$ and $b   \tdiv a$, then $a = b$ or $a = -b$
    \item If $a  \tdiv b$ and $a  \tdiv c$, then $a   \tdiv (n \cdot b + m \cdot c)$ for all integers $n$ and $m$
\end{itemize}

\section{Modular Arithmetic}
Modular arithmetic is a system of arithmetic for integers, which considers the remainder. In modular arithmetic, numbers ``wrap-around'' a given integer to leave a remainder. This branch of mathematics appears in computer science, computer algebra, and cryptography.

\subsection{Modulus}
Given two positive integers $a$ and $b$, $a$ \textit{modulo} $b$ ($a \bmod b$), is the remainder when $a$ is divided by $b$. \\

\textbf{Examples:}
\begin{align*}
    10 \bmod 5 = 0 && 32 \bmod 6 = 2 && 61 \bmod 12 = 1
\end{align*}

\subsection{Congruence}
For a positive integer $n$, integers $a$ and $b$ are said to be \textit{congruent} modulo $n$, if their remainders when divided by $n$ are the same. $$a \equiv b \pmod n$$
\textbf{Examples:}
\begin{align*}
    12 \equiv 0 \pmod 2 && 9 \equiv 3 \pmod 6 && 20 \equiv 10 \pmod 5
\end{align*}

\subsection{Properties}
\subsubsection{Addition}
\begin{itemize}
    \item If $a + b = c$, then $a \bmod N + b \bmod N \equiv c \bmod N$
    \item If $a \equiv b \pmod N$, then $a + k \equiv b + k \pmod N$, for all integers $k$
    \item If $a \equiv b \pmod N$ and $c \equiv d \pmod N$, then $a + c \equiv b + d \pmod N$
    \item If $a \equiv b \pmod N$, then $-a \equiv -b \pmod N$
\end{itemize}

\subsubsection{Multiplication}
\begin{itemize}
    \item If $a \cdot b = c$, then $a \bmod N \cdot b \bmod N \equiv c \bmod N$
    \item If $a \equiv b \pmod N$, then $ka \equiv kb \pmod N$, for all integers $k$
    \item If $a \equiv b \pmod N$ and $c \equiv d \pmod N$, then $ac \equiv bd \pmod N$
\end{itemize}

\subsubsection{Exponentiation}
\begin{itemize}
    \item If $a \equiv b \pmod N$, then $a^k \equiv b^k \pmod N$ for all integers $k$
\end{itemize}

\end{document} % Need to add a note about negative mod - not the same as computer science mod