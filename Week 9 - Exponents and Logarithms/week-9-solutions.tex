\documentclass[12pt]{article}
\usepackage[a4paper, total={7in, 10in}]{geometry}
\usepackage{amsmath}
\usepackage{graphicx}

\title{Week 9 Problem Set Solutions \vspace{-3mm}}
\author{2018-2019 SLSS Math Club\vspace{-5mm}}
\date{December 18, 2018\vspace{-5mm}}

\begin{document}
\maketitle

\section*{Basic Problems}
\begin{enumerate}
    \item \textbf{Determine $x$ such that $\log_x2 + \log_x4 + \log_x8 = 1$.} \\
    
    Applying our logarithm and exponent laws, we can solve for $x$ as follows:
    \begin{align*}
        1 &= \log_x2 + \log_x4 + \log_x8 \\
        1 &= \log_x64 \\
        x^1 &= 64 \\
        x &= 64
    \end{align*}
    
    Therefore, $x = 64$.
    
    \item \textbf{If $\displaystyle{x^3y^5 = 2^{11}  3^{13}}$ and $\displaystyle{\frac{x}{y^2} = \frac{1}{27}}$, solve for $x$ and $y$.} \\
    
    Applying our exponent laws to rearrange the second equation, we yield the following:
    \begin{align*}
        \frac{x}{y^2} &= \frac{1}{27} \\
        xy^{-2} &= 3^{-3} \\
        x^3 y^{-6} &= 3^{-9} 
    \end{align*}
    
    Dividing the the first equation by our second equation, we can solve for $y$:
    \begin{align*}
        \frac{x^3y^5}{x^3y^{-6}} &= \frac{2^{11}3^{13}}{3^{-9}} \\
        y^{11} &= 2^{11} 3^{22} \\
        y &= 2(3^2) \\
        y &= 18
    \end{align*}
    
    Solving for $x$:
    \begin{align*}
        \frac{x}{y^2} &= \frac{1}{27} \\
        \frac{x}{18^2} &= \frac{1}{27} \\
        x &= 12
    \end{align*}
    
    Therefore, $x = 12$ and $y = 18$.
    
\end{enumerate} \newpage

\section*{Intermediate Problems}
\begin{enumerate}
    \item \textbf{Determine the sum of the following series:
    \begin{equation*}
        \log_{10}\frac{3}{2} + \log_{10}\frac{4}{3} + \log_{10}\frac{5}{4} + \dots + \log_{10}\frac{200}{199}
    \end{equation*}} \\
    
    We know that $$\log_x(a) + \log_x(b) = \log_x(ab)$$ As such, we can solve for the sum as follows:
    \begin{align*}
        S &= \log_{10}\frac{3}{2} + \log_{10}\frac{4}{3} + \log_{10}\frac{5}{4} + \dots + \log_{10}\frac{200}{199} \\
        S &= \log_{10}[\frac{(3)(4)(5)\dots(200)}{(2)(3)(4)\dots(199)}] \\
        S &= \log_{10}(\frac{200}{2}) \\
        S &= \log_{10}(100) \\
        S &= 2
    \end{align*}
    
    Therefore, the sum of the series is $2$.
    
    \item \textbf{Solve the equations for the point of intersection of the graphs of $y = \log_2(2x)$ and $y = \log_4(x)$.} \\
    
    To solve for the $x$ component of the point of intersection, we set the two equations equal to each other:
    \begin{align*}
        \log_2(2x) &= \log_4(x) \\
        \log_2(2x) &= \frac{1}{2}\log_2(x) \\
        1 + \log_2(x) &= \frac{1}{2}\log_2(x) \\
        \frac{1}{2}\log_2(x) &= -1 \\
        \log_2(x) &= -2 \\
        x &= 2^{-2} \\
        x &= \frac{1}{4}
    \end{align*}
    
    By substituting $x = \frac{1}{4}$, we can solve for $y$:
    \begin{align*}
        y &= \log_2(2x) \\
        y &= \log_2[2(\frac{1}{4})] \\
        y &= \log_2(\frac{1}{2}) \\
        y &= -1
    \end{align*}
    
    Therefore, the point of intersection between the two graphs is $(\frac{1}{4}, -1)$.
    
\end{enumerate} \newpage

\section*{Advanced Problems}
\begin{enumerate}
    \item \textbf{Prove that $a, b,$ and $c$ form a geometric sequence if and only if $\log_x{a}, \log_x{b},$ and $\log_x{c}$ form an arithmetic sequence.} \\
    
    As $a, b, c$ forms a geometric sequence, we can state that:
    \begin{align*}
        a &= a \\
        b &= ar \\
        c &= ar^2
    \end{align*}
    
    
    As such,
    \begin{align*}
        \log_x(a) &= \log_x(a) \\
        \log_x(b) &= \log_x(ar) \\
                  &= \log_x(a) + \log_x(r) \\
        \log_x(c) &= \log_x(ar^2) \\
                  &= \log_x(a) + 2\log_x(r)
    \end{align*}
    
    Therefore, as shown, $a, b,$ and $c$ form a geometric sequence if and only if $\log_x{a}, \log_x{b},$ and $\log_x{c}$ form a arithmetic sequence.
    
    \item \textbf{$\displaystyle{ 2^{x + 3} + 2^x = 3^{y + 2} - 3^y}$ and $x$ and $y$ are integers, determine the values of $x$ and $y$.} \\
    
    Note: Since $x$ and $y$ are integers, $2^x$ and $3^y$ will always be powers of $2$ and $3$, respectively. \\
    
    By rearranging the equations, and applying the above property, we can solve for $x$ and $y$:
    \begin{align*}
        2^{x + 3} + 2^x &= 3^{y + 2} - 3^y \\
        2^x(2^3 + 1) &= 3^y(3^2 - 1) \\
        2^x (9) &= 3^y (8) \\
        2^x 3^2 &= 2^3 3^y 
    \end{align*}
    
    Therefore, $x = 3$ and $y = 2$.
    
\end{enumerate}
\end{document}
