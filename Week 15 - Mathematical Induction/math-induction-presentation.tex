\documentclass{beamer}
\usepackage[utf8]{inputenc}
\usetheme{Antibes}
\usecolortheme{beaver}
\setbeamertemplate{navigation symbols}{}

\title{Mathematical Induction Lesson}
\author{2018-2019 SLSS Math Club}
\date{May 15, 2019}

\begin{document}
\frame{\titlepage}

\section{Introduction}
\begin{frame}{Introduction}
   Mathematical Induction is a method of proof which is used when proving something is true for integers.
   This method can be divided into 3 steps:
   \begin{enumerate}
       \item \textbf{Base Case:} Show that the desired result holds for a specific value (eg. $n = 1$).
       \item \textbf{Inductive Hypothesis:} Assume that the result holds when $n = k$.
       \item \textbf{Inductive Step:} Using the inductive hypothesis, show that the result holds when $n = k + 1$. In other words show that having the result hold for $n = k$ implies that the result holds for $n = k + 1$.
   \end{enumerate}
\end{frame}

\section{Reasoning}
\begin{frame}{Reasoning Behind The Proof}
    Essentially, the inductive step shows that if some the desired result holds for $n = k$, then the desired result will also hold for $n = k + 1$. This step alone is not enough to complete the proof. It is essential that the desired result also holds for some base case so that the result will also hold for all integers greater than the base case. \newline
    
    The following metaphor helps explain this concept:
    \begin{quote}
        Mathematical induction proves that we can climb as high as we like on a ladder, by proving that we can climb onto the bottom rung (the basis) and that from each rung we can climb up to the next one (the step). \newline
        --- Concrete Mathematics.
    \end{quote}
\end{frame}

\section{Examples}
\begin{frame}{Sum of Natural Numbers}
    Show that the following assertion is correct:
    $$1 + 2 + 3 + ... + n = \frac{n(n+1)}{2}$$
    
    \textbf{Base Case:}
    First, we show that it is true when $n = 1$. 
    $$1 = \frac{1(1+1)}{2}$$
    
    \textbf{Inductive Hypothesis}:
    If the assertion is true for k, then
    $$1 + 2 + 3 + ... + k = \frac{k(k+1)}{2}$$.
\end{frame}

\begin{frame}{Sum of Natural Numbers}
    \textbf{Inductive Step:}
    Assuming the inductive hypothesis is true, we show that the assertion is also true for k+1:
    \begin{align*}
        1 + 2 + 3 + ... + k + (k+1) \\
        &= (1 + 2 + 3 + ... + k) + (k+1) \\
        &= \frac{k(k+1)}{2} + (k+1) \\
        &= \frac{k(k+1)+2(k+1)}{2} \\
        &= \frac{(k+1)(k+2)}{2}
    \end{align*}
    
    This proves the inductive step and the induction is complete.
\end{frame}

\begin{frame}{Inequality}
    Prove by induction that $2^n > 2n$ for every positive integer $n > 2$. \newline
    
    \textbf{Proof:} For $n = 3$, we have $2^3 > 2(3)$ or $8 > 6$, so the inequality is correct for $n = 3$. \newline
    If the assertion is true for $n = k$, we have $2^k > 2k$. \newline
    Thus,
    \begin{align*}
        2 \cdot 2^k &> 2 \cdot 2k \\
        2^{k+1} &> 4k \\
        2^{k+1} &> 2(k+1)
    \end{align*}
    
    This proves the inductive step and the proof is complete.
\end{frame}

\end{document}